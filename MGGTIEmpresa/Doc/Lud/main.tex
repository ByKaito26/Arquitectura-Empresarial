% Aquí empieza la parte del documento
% Una parte muy interesante.
% Buenas tardes, hasta luego.

\section{Análisis de la situación actual de la Bodytech}

Para hacer un análisis de la situación actual de Bodytech, se van a tomar en cuenta los documentos proporcionados por el MinTIC; \textit{Dominio de Estrategia de TI}\cite{mintic_estrategia_2022}, \textit{Dominio de Gobierno de TI}\cite{mintic_gobierno_2022}, \textit{Dominio de Gestión de Sistemas de Información} \cite{mintic_sistemas_2022}, \textit{Dominio de Gestión de la Información}\cite{mintic_informacion_2022}, \textit{Dominio de Gestión de Servicios de TI}\cite{mintic_servicios_2022}, y, \textit{Dominio de Uso y Apropiación de TI}\cite{mintic_apropiacion_2022}. Teniendo estos documentos en cuenta, se pueden realizar evaluaciones al estado actual de la empresa de acuerdo a cada documento:

\subsection{Estrategia de TI}

Bodytech ha demostrado una fuerte orientación hacia la transformación digital como soporte de su modelo de negocio. Su presencia en varios países latinoamericanos ha exigido una estrategia tecnológica coherente y escalable. La digitalización de sus servicios responde a la necesidad de ofrecer experiencias omnicanal a sus clientes, mejorar la fidelización y mantenerse competitivo frente a nuevas tendencias como el fitness digital y la medicina personalizada.

\begin{itemize}
  \item Bodytech ha hecho inversiones considerables en plataformas digitales como apps móviles, portales web y herramientas de monitoreo de actividad física.
  \item Su estrategia TI parece alineada con su visión de ``ser la red médico-deportiva líder en América Latina''.
  \item Utiliza la tecnología como habilitador estratégico para fidelizar clientes, mejorar la experiencia y ampliar servicios virtuales (telemedicina, entrenamientos remotos, etc.).
\end{itemize}

\subsection{Gobierno de TI}

Actualmente, Bodytech parece manejar la toma de decisiones de TI de manera centralizada, lo cual puede ser eficiente en ciertas etapas, pero podría limitar la agilidad y transparencia en el gobierno de TI. La existencia de comités, políticas claras y estructuras de rendición de cuentas sería clave para garantizar que las decisiones de inversión, riesgo y alineación estratégica estén bien sustentadas.

\begin{itemize}
  \item La toma de decisiones sobre tecnología parece estar centralizada en la alta dirección, pero no se conoce si hay un comité de TI o un ente que evalúe riesgos, inversiones y resultados en TI.
  \item Probable uso de proveedores externos para servicios TI (tercerización), lo cual requiere mecanismos de control claros.
\end{itemize}

\subsection{Gestión de Sistemas de Información}

El ecosistema digital de Bodytech debe soportar una operación altamente demandante y geográficamente distribuida. La correcta integración de sistemas como CRM, ERP, historia clínica y apps móviles es fundamental para ofrecer un servicio fluido y personalizado. Sin embargo, el crecimiento acelerado puede generar problemas de interoperabilidad y obsolescencia de algunos módulos tecnológicos.

\begin{itemize}
  \item Bodytech cuenta con un ecosistema digital: CRM, ERP, plataformas de reservas, aplicaciones móviles, y posiblemente sistemas para historia clínica deportiva.
  \item Es clave la interoperabilidad entre sistemas para personalizar la atención y dar seguimiento a la salud y desempeño físico del usuario.
\end{itemize}

\subsection{Gestión de la Información}

En un entorno tan sensible como el del fitness médico-deportivo, la gestión de datos adquiere un valor crítico. Bodytech debe garantizar la seguridad, privacidad, calidad y disponibilidad de la información, especialmente considerando que maneja datos personales, clínicos y de comportamiento. La analítica avanzada representa una gran oportunidad, pero también un reto ético y técnico.

\begin{itemize}
  \item Bodytech maneja grandes volúmenes de datos personales, médicos y de comportamiento del usuario. Esto requiere gobernanza fuerte.
  \item Probable uso de analítica para personalización de rutinas y marketing segmentado.
\end{itemize}

\subsection{Gestión de Servicios de TI}

Los servicios digitales ofrecidos por Bodytech requieren una infraestructura robusta, con alta disponibilidad y soporte constante. La calidad del servicio TI impacta directamente en la satisfacción del cliente. Es necesario contar con procesos claros para la gestión de incidentes, niveles de servicio (SLAs) y monitoreo continuo del desempeño.

\begin{itemize}
  \item Cuentan con aplicaciones móviles y atención virtual, lo que implica gestión de infraestructura en la nube, disponibilidad de servicios y soporte.
  \item La experiencia del usuario depende en gran medida de la estabilidad y funcionalidad del servicio TI.
\end{itemize}

\subsection{Uso y Apropiación de TI}

Aunque Bodytech ha desarrollado plataformas intuitivas y servicios digitales atractivos, no todos los usuarios tienen el mismo nivel de apropiación tecnológica. Existe una brecha digital que puede afectar la experiencia del cliente y la efectividad de los servicios. Por ello, es importante implementar estrategias de capacitación digital y accesibilidad.

\begin{itemize}
  \item Bodytech promueve el uso activo de sus plataformas (app, entrenadores virtuales, seguimiento de progreso).
  \item Probable brecha digital entre distintos segmentos de usuarios (por edad, habilidades digitales o acceso a tecnología).
\end{itemize}


Después de esta evaluación podemos decir que Bodytech ha avanzado notablemente en la integración de la tecnología como habilitador de su propuesta de valor. Sin embargo, la madurez en Gobierno y Gestión de TI puede fortalecerse sistemáticamente con base en el MGGTI.

\section{Diagnóstico e identificación de brechas}

Para cada uno de los seis dominios del MGGTI se realiza un diagnóstico actual, un nivel estimado de madurez (bajo, medio, alto) y las brechas encontradas.

\subsection{Dominio: Estrategia de TI}

\textbf{Diagnóstico:} Bodytech alinea sus iniciativas tecnológicas con su visión de negocio. Dispone de aplicaciones móviles, servicios virtuales y un enfoque hacia la transformación digital.

\textbf{Madurez estimada:} Medio

\textbf{Brechas identificadas:}
\begin{itemize}
  \item Falta de un Plan Estratégico de TI formalizado y documentado.
  \item Ausencia de indicadores de desempeño TI estratégicos (KPIs).
  \item No se evidencian mapas de ruta tecnológicos alineados a mediano y largo plazo.
\end{itemize}

\subsection{Dominio: Gobierno de TI}

\textbf{Diagnóstico:} La toma de decisiones TI parece concentrada en la alta gerencia sin una estructura formalizada de gobierno.

\textbf{Madurez estimada:} Bajo - Medio

\textbf{Brechas identificadas:}
\begin{itemize}
  \item Ausencia de un Comité de Gobierno de TI institucionalizado.
  \item No hay evidencia de roles y responsabilidades claramente definidos bajo esquemas como RACI.
  \item No se aplican marcos internacionales de gobierno como COBIT o ISO/IEC 38500.
\end{itemize}

\subsection{Dominio: Gestión de Sistemas de Información}

\textbf{Diagnóstico:} Existen múltiples sistemas de información (app móvil, software de gestión, CRM, etc.) que aportan valor al servicio.

\textbf{Madurez estimada:} Medio

\textbf{Brechas identificadas:}
\begin{itemize}
  \item No se ha implementado un portafolio institucional de sistemas con evaluación periódica.
  \item Riesgos de sistemas aislados o con baja interoperabilidad.
  \item Se desconoce si hay procesos formales de adquisición, evaluación y retiro de tecnologías.
\end{itemize}

\subsection{Dominio: Gestión de la Información}

\textbf{Diagnóstico:} Bodytech maneja información sensible de clientes (datos médicos, rutinas, preferencias, historial físico).

\textbf{Madurez estimada:} Medio

\textbf{Brechas identificadas:}
\begin{itemize}
  \item Ausencia de una política de gobernanza de datos.
  \item No se evidencia un catálogo institucional de datos.
  \item Se requiere mejorar la calidad, seguridad y analítica de los datos bajo estándares éticos y regulatorios.
\end{itemize}

\subsection{Dominio: Gestión de Servicios de TI}

\textbf{Diagnóstico:} Disponen de infraestructura digital activa (apps, web, soporte) que da soporte a sus clientes y operaciones.

\textbf{Madurez estimada:} Medio

\textbf{Brechas identificadas:}
\begin{itemize}
  \item No se identifican catálogos formales de servicios TI.
  \item Ausencia de procesos tipo ITIL para la gestión del ciclo de vida de servicios.
  \item Se requiere mejorar la medición de experiencia del usuario (UX) y niveles de servicio (SLA).
\end{itemize}

\subsection{Dominio: Uso y Apropiación de TI}

\textbf{Diagnóstico:} Alto uso por parte de usuarios avanzados, pero hay brechas de apropiación tecnológica en segmentos específicos.

\textbf{Madurez estimada:} Medio - Alto

\textbf{Brechas identificadas:}
\begin{itemize}
  \item Falta de estrategias para fortalecer la alfabetización digital de usuarios menos hábiles.
  \item Poca evidencia de procesos para evaluar la percepción y apropiación tecnológica.
  \item Se pueden mejorar mecanismos de retroalimentación para ajustar funcionalidades de las plataformas digitales.
\end{itemize}

\subsection*{Resumen}

\begin{table}[H]
  \centering
  \begin{tabular}{|p{5cm}|c|p{7cm}|}
    \hline
    \textbf{Dominio} & \textbf{Madurez} & \textbf{Brecha Principal} \\
    \hline
    Estrategia de TI & Medio & Plan estratégico formal y KPIs \\
    \hline
    Gobierno de TI & Bajo - Medio & Comité y políticas de gobernanza \\
    \hline
    Gestión de Sistemas de Información & Medio & Portafolio institucional y evaluación \\
    \hline
    Gestión de la Información & Medio & Política de gobernanza de datos \\
    \hline
    Gestión de Servicios de TI & Medio & Catálogo de servicios y SLA \\
    \hline
    Uso y Apropiación de TI & Medio - Alto & Inclusión digital y evaluación de uso \\
    \hline
  \end{tabular}
  \caption{Resumen del diagnóstico por dominio del MGGTI aplicado a Bodytech}
\end{table}
