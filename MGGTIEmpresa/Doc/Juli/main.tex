% Aquí empieza la parte del documento
% Una parte muy interesante.
% Buenas tardes, hasta luego.
\section{Objetivos Estratégicos}

La implementación del Modelo de Gestión y Gobierno de TI (MGGTI) en Bodytech tiene como propósito fundamental fortalecer la gestión estratégica, operativa y táctica de las tecnologías de la información, alineándolas directamente con los objetivos del negocio. En este sentido, se definen los siguientes objetivos estratégicos:

\begin{itemize}
    \item \textbf{Alinear las capacidades tecnológicas con la visión corporativa de Bodytech}, promoviendo una transformación digital robusta, escalable y sostenible.
    \item \textbf{Establecer un gobierno de TI eficiente y transparente}, que facilite la toma de decisiones basadas en datos, riesgos y beneficios tecnológicos claros.
    \item \textbf{Mejorar la interoperabilidad de los sistemas de información existentes}, optimizando la experiencia del cliente y garantizando continuidad operativa en todas las sedes de Latinoamérica.
    \item \textbf{Fortalecer la gestión de datos sensibles y clínicos}, implementando políticas de gobernanza de información bajo estándares éticos y normativos.
    \item \textbf{Estandarizar la gestión de servicios de TI}, incorporando buenas prácticas como ITIL para elevar la calidad del servicio y la satisfacción del usuario.
    \item \textbf{Reducir la brecha de apropiación tecnológica}, mediante estrategias de capacitación, accesibilidad y retroalimentación de usuarios.
\end{itemize}

\section{Alcance del Proyecto}

El presente plan de implementación del MGGTI contempla la adopción progresiva y sistémica de los seis dominios que lo componen, con un enfoque integral que involucra tanto áreas técnicas como estratégicas de Bodytech. El alcance se define de la siguiente manera:

\begin{enumerate}[label=\textbf{\arabic*.}]
    \item \textbf{Dominio de Estrategia de TI:} Elaboración de un Plan Estratégico de TI formal, que contemple mapas de ruta tecnológicos, KPIs y mecanismos de seguimiento a mediano y largo plazo.
    
    \item \textbf{Dominio de Gobierno de TI:} Diseño e institucionalización de un Comité de Gobierno de TI, definición de roles y responsabilidades bajo el modelo RACI, y adopción de marcos como COBIT o ISO/IEC 38500.
    
    \item \textbf{Dominio de Gestión de Sistemas de Información:} Creación de un portafolio institucional de sistemas, políticas de evaluación y renovación tecnológica, e integración plena entre sistemas CRM, ERP, apps móviles e historia clínica.
    
    \item \textbf{Dominio de Gestión de la Información:} Implementación de una política de gobernanza de datos, catálogo de datos institucional, y mejora continua en calidad, seguridad y analítica de la información.
    
    \item \textbf{Dominio de Gestión de Servicios de TI:} Formalización de un catálogo de servicios, adopción del ciclo de vida de servicios bajo ITIL, establecimiento de SLAs y mejora de la experiencia de usuario mediante métricas UX.
    
    \item \textbf{Dominio de Uso y Apropiación de TI:} Desarrollo de estrategias de alfabetización digital, segmentadas según perfiles de usuario, y diseño de procesos continuos de retroalimentación para mejorar las plataformas digitales.
\end{enumerate}

\bigskip

Este plan se desarrollará en fases, priorizando la estabilización del gobierno de TI y la estrategia tecnológica, y luego abordando la gestión operativa, informacional y de servicio, con indicadores claros de avance y madurez en cada etapa. La implementación se considera clave para consolidar a Bodytech como una red médico-deportiva líder e innovadora en América Latina.
