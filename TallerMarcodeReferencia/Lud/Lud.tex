% Aquí va el contenido que hace Lud, recordar no agregar preámbulos ni importar paquetes en este archivo.
% El código que se va modificando se puede compilar usando el script Run.sh y visualizar el PDF que está
% en el directorio de MainDoc bajo el nombre de main.pdf

\section*{Análisis de brechas estratégicas y tecnológicas}

Con base en el análisis de la situación actual del \textbf{Ministerio de Innovación y Desarrollo Digital (MIDD)} y las brechas identificadas, se realiza una priorización de acuerdo con su impacto estratégico y tecnológico.

\begin{itemize}
    \item \textbf{Gobernanza de TI} \\
    \textit{Brecha: Ausencia de liderazgo técnico y estratégico} \\
    Esta brecha representa una prioridad estratégica crítica. Sin una gobernanza clara, con funciones y roles definidos, es inviable coordinar iniciativas tecnológicas de gran escala. \\
    \textbf{Recomendación:} Establecer una oficina de arquitectura empresarial y liderazgo técnico transversal para alinear TI con los objetivos institucionales.

    \item \textbf{Infraestructura} \\
    \textit{Brecha: Falta de inversión y rigidez tecnológica} \\
    Se trata de una prioridad tecnológica alta. La infraestructura obsoleta limita seriamente la escalabilidad, agilidad y seguridad de los servicios digitales. \\
    \textbf{Recomendación:} Migrar gradualmente a infraestructuras híbridas, aprovechando servicios en la nube y mecanismos de modernización continua.

    \item \textbf{Automatización} \\
    \textit{Brecha: Ineficiencia operativa y alta carga manual} \\
    Esta brecha tiene un impacto importante en la eficiencia operativa. La falta de automatización frena la productividad, aumenta el error humano y reduce la capacidad de adaptación. \\
    \textbf{Recomendación:} Incorporar flujos de trabajo inteligentes e inteligencia artificial (IA) en procesos críticos.

    \item \textbf{Interoperabilidad} \\
    \textit{Brecha: Dificultad para integrar nuevos sistemas} \\
    Tiene un impacto medio-alto en la arquitectura tecnológica. La falta de integración entre sistemas limita la colaboración y el aprovechamiento de los datos. \\
    \textbf{Recomendación:} Adoptar una arquitectura modular basada en microservicios que permita escalabilidad e integración ágil.

    \item \textbf{Gestión del cambio} \\
    \textit{Brecha: Baja apropiación tecnológica interna} \\
    Si bien su impacto no es inmediato desde el punto de vista técnico, esta brecha afecta la sostenibilidad de las transformaciones tecnológicas. \\
    \textbf{Recomendación:} Promover una cultura digital mediante embajadores del cambio, capacitaciones continuas y reconocimiento a buenas prácticas.
\end{itemize}

La transformación digital del MIDD debe iniciar con acciones urgentes en las áreas de \textbf{gobernanza de TI} e \textbf{infraestructura tecnológica}, ya que estas constituyen la base habilitadora de los procesos de automatización, interoperabilidad y gestión del cambio. La falta de liderazgo claro y de tecnologías modernas puede comprometer la sostenibilidad y escalabilidad de los esfuerzos institucionales hacia la innovación.

