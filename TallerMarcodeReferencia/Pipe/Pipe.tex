\subsection{Innovación y Creatividad en la Solución}

La propuesta de arquitectura de TI para el MIDD incluye soluciones innovadoras que responden a los desafíos actuales y anticipan futuras necesidades tecnológicas. Se destacan los siguientes elementos innovadores:

\begin{itemize}
  \item \textbf{Plataforma unificada de gestión de TI basada en microservicios}: que permite la integración flexible de nuevas funcionalidades y asegura la escalabilidad del sistema.
  \item \textbf{Uso de inteligencia artificial (IA) para la automatización de procesos operativos}, incluyendo mantenimiento predictivo de sistemas y asignación inteligente de recursos tecnológicos.
  \item \textbf{Dashboard central de gobernanza TI}, que facilita el seguimiento en tiempo real de indicadores clave de desempeño (KPIs), seguridad y disponibilidad de servicios digitales.
  \item \textbf{Modelo híbrido de infraestructura}, combinando soluciones en la nube pública con recursos locales estratégicos, lo que garantiza mayor control sobre datos sensibles y mayor resiliencia operativa.
  \item \textbf{Catálogo digital de servicios de TI}, que facilita la consulta, solicitud y monitoreo de los servicios disponibles para todas las áreas del ministerio, fomentando la transparencia y el autoservicio.
\end{itemize}

Estas innovaciones buscan fortalecer la eficiencia, reducir costos, aumentar la transparencia y promover la adopción de nuevas tecnologías en el MIDD.

\subsection{Alineación con la Política de Gobierno Digital}

La propuesta está en línea con la Política de Gobierno Digital del Estado colombiano, especialmente en los siguientes puntos:

\begin{itemize}
    \item \textbf{Servicios ciudadanos digitales:} La arquitectura propuesta optimiza los servicios institucionales para garantizar accesibilidad y eficiencia.

    \item \textbf{Gestión estratégica de TI:} Se implementa un enfoque centrado en la gobernanza y el valor público, integrando la tecnología con las necesidades de los ciudadanos y del sector público.

    \item \textbf{Datos como activos estratégicos:} Se impulsa la gestión adecuada de los datos mediante estándares de interoperabilidad y analítica avanzada.

    \item \textbf{Seguridad digital:} Se proponen políticas de ciberseguridad alineadas con estándares nacionales e internacionales.
\end{itemize}

\section{Análisis de Brechas y Hoja de Ruta}

\subsection{Brechas Encontradas}

\begin{table}[ht]
    \centering
    \begin{tabular}{|p{0.2\textwidth}|p{0.25\textwidth}|p{0.25\textwidth}|p{0.25\textwidth}|}
        \hline
        \textbf{Dimensión} & \textbf{Situación actual} & \textbf{Situación objetivo} & \textbf{Brecha identificada} \\ \hline
        Infraestructura & Sistemas heredados, infraestructura obsoleta & Infraestructura híbrida, basada en la nube, con capacidad de escalabilidad y modernización continua & Falta de inversión, rigidez tecnológica \\ \hline
        Gobernanza de TI & Procesos desarticulados, sin roles definidos & Modelo de gobernanza con funciones, responsabilidades y métricas claras & Ausencia de liderazgo técnico y estratégico \\ \hline
        Automatización & Baja automatización de procesos TI & Procesos automatizados con IA y flujos inteligentes de trabajo & Ineficiencia operativa y alta carga manual \\ \hline
        Interoperabilidad & Sistemas aislados, sin integración & Arquitectura modular y basada en microservicios & Dificultad para integrar nuevos sistemas \\ \hline
        Gestión del cambio & Resistencia a la adopción tecnológica & Cultura digital fortalecida y capacitaciones continuas & Baja apropiación tecnológica interna \\ \hline
    \end{tabular}
    \caption{Análisis de brechas para el Ministerio de Innovación y Desarrollo Digital (MIDD)}
    \label{tab:p3-1}
\end{table}





