\section{Desarrollo de la Hoja de Ruta}

\subsection{Estructura y claridad de la hoja de ruta}

La hoja de ruta para la implementación de la nueva arquitectura de TI en el MIDD está organizada en fases estratégicas, con iniciativas claramente definidas, cronograma estimado y responsables asignados, de la siguiente manera:

\begin{center}
\begin{tabular}{|p{3cm}|p{5cm}|p{3cm}|p{3cm}|}
\hline
\textbf{Fase} & \textbf{Iniciativas Clave} & \textbf{Tiempo Estimado} & \textbf{Responsables} \\
\hline
Diagnóstico & Levantamiento de infraestructura actual, análisis de sistemas y procesos & 1 mes & Equipo de TI + Consultores externos \\
\hline
Diseño & Definición de la arquitectura de TI, selección de tecnologías, diseño de APIs e integración & 2 meses & Arquitecto de TI + Jefe de proyectos \\
\hline
Piloto & Implementación de módulos clave en entorno controlado, pruebas de integración & 2 meses & Desarrolladores + Usuarios clave \\
\hline
Despliegue & Escalamiento progresivo del sistema, migración de datos y usuarios & 3 meses & Equipo de TI + Líderes de áreas \\
\hline
Seguimiento y mejora continua & Evaluación de desempeño, ajustes, formación continua & Permanente & Oficina de TI + Dirección General \\
\hline
\end{tabular}
\end{center}

\vspace{1em}

\subsection{Viabilidad de las iniciativas propuestas}

Las actividades y proyectos propuestos se han diseñado considerando el contexto organizacional y presupuestal del MIDD, así como la disponibilidad de talento humano. Las tecnologías sugeridas (como arquitecturas en la nube híbrida, APIs REST, microservicios y herramientas de automatización) son ampliamente utilizadas en entornos públicos y permiten una implementación modular y escalable. Se contempla además el uso de software libre y estándares abiertos para reducir costos y garantizar la interoperabilidad. La adopción por fases minimiza riesgos y permite generar valor progresivo.

\vspace{1em}

\subsection{Estrategia de gestión del cambio}

Se plantea una estrategia de gestión del cambio centrada en el involucramiento de los grupos de interés internos desde las primeras fases del proyecto. Esta estrategia incluye:

\begin{itemize}
    \item \textbf{Comunicación constante:} sesiones informativas, boletines internos y canales abiertos para resolver dudas.
    \item \textbf{Formación y capacitación:} programas dirigidos a usuarios finales y administradores de sistemas.
    \item \textbf{Identificación de líderes de cambio:} personal de cada área que actúe como embajador de la transformación.
    \item \textbf{Gestión de expectativas:} priorización de necesidades reales, evaluación de impacto y retroalimentación constante.
    \item \textbf{Acompañamiento post-despliegue:} soporte técnico, ajustes y espacios de escucha activa.
\end{itemize}

Con estas acciones, se busca lograr una transición ordenada, aumentar la adopción de la nueva arquitectura y asegurar que el cambio tecnológico se traduzca en mejoras reales para la operación institucional.
